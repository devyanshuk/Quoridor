\chapter*{Introduction}
\addcontentsline{toc}{chapter}{Introduction}

Artificial Intelligence has brought a revolutionary transformation to the gaming world, pushing the boundaries
of what's achievable in both single and multiplayer gaming experiences. In recent years, AI has taken center
stage with its remarkable accomplishments in mastering age-old games such as Chess, Go, and many others.
AI-driven games now offer users the opportunity to hone and enhance their skills, providing varying difficulty
levels and offering optimal moves to guide players through each step if desired.

Designed by Mirko Marchesi, Quoridor stands out as an engaging and strategic board game that pits two or four
players against each other in a thrilling race to traverse a maze-like board and reach the opposite side.
Played on a chess-like board, this game introduces a fascinating twist with walls that players strategically
place to obstruct their opponents' paths, compelling them to navigate longer alternate routes. Despite its
seemingly simple rules, Quoridor demands a unique blend of strategic foresight and the ability to anticipate
the moves of opponents. It's a dynamic and thought-provoking game that challenges the mind.

The primary objective of this thesis is to construct a well-structured framework and user-friendly interfaces
that seamlessly integrate AI algorithms into the Quoridor game. The development of AI algorithms customized to
Quoridor's unique rule set will not only enhance our understanding of the game's intricate nuances but also
facilitate the creation of an intuitive interface for simulating these AI agents. Furthermore, a comprehensive
evaluation will be conducted to identify the top-performing AI agent.

In addition to this, the project will encompass the creation of a user-friendly interface that empowers players
to engage with an AI opponent of their choice, thereby bolstering the game's accessibility and inclusivity.

\section {Acknowledged Works}

Quoridor, being a widely popular game, has attracted a fair number of attention from the research community,
resulting in successful AI agent developments. Some of the notable works include:

\begin{itemize}
    \item \textbf{Mastering Quoridor \citep{Glendenning2002MasteringQ}}
    The writer implements and assesses various algorithms like Negamax, Alpha-beta negamax among others.
    Additionally, they utilized a genetic algorithm to refine the weights within a linear weighted evaluation
    function, employing 10 distinct features suggested by the author, some of which include player's position
    towards their goal side, the opponent's position towards their respective goal, the remaining count of
    walls available to the player, etc.
\end{itemize}


In sharp contrast to the aforementioned paper, our thesis takes a distinctly different path by delving deeply
into the architectural aspects of AI. Our approach emphasizes abstraction to the greatest extent possible, with
an eye on facilitating seamless integration into a broad spectrum of games. We prioritize creating an
interface that is adaptable to diverse game environments, setting our research apart from the game-specific
focus of the prior works.



