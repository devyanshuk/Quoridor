\chapter{Introduction}

In recent years, \gls{AI} is becoming an integral part of many elements of modern world including gaming \citep{Skinner2010Artificial}, pushing the boundaries of what's achievable in both single and multiplayer gaming experiences. \gls{AI}-driven games now offer users the opportunity to hone and enhance their skills, providing varying difficulty levels and offering optimal moves to guide players through each step if desired. \gls{AI} has also taken a center stage in gaming with its remarkable accomplishments in age-old strategy games such as Chess, Go, and many others. 

% . In fact, in the recent years, it has become an the main point of evolution and revolution in many of the technologies and industries. From assisted or autonomous driving \citep{Ma2020Artificial}, chat bots \citep{Wu2023ABrief} to gaming \citep{Skinner2010Artificial}, it has been a major revelation in evolving the existing technologies to generating new industry space with the plethora of new use cases.   
% 

Strategy games are a genre of gaming that require planning, often involving various tactics, decision making and execution under various resource constraints. Some examples of strategy games include Shogi, Starcraft and Quoridor. They are unique compared to other genres as they require a selection of an optimal move among multiple possible moves based on a certain strategy. In many scenarios, the number of possible moves depends on the game tree, simulation and prediction of the player's and the opponent's moves, all while managing resources efficiently.

\gls{AI}, due to its suitability of solving complex decision making problems factoring in multiple variable and constraints, has been particularly effective in playing the strategy games. The history of \gls{AI} in strategy gaming dates few decades. One of the oldest marked impact of \gls{AI} in the strategy gaming came in 1997 when IBM's Deep Blue \citep{Campbell2002Deep} defeated World Chess Champion Garry Kasparov. The influence was more prominent with the success of \gls{AI} on \gls{RTS} games such as Warcraft and StarCraft \citep{Robertson2014Review} and strategy games such as Go \citep{Huang2011Monte}. Recently, DeepMind's Alpha Go for Go, Alpha Zero for Chess \citep{Silver2017Mastering} and AlphaStar for StarCraft \citep{Team2019Alphastar} have widened the gap between the \gls{AI} and human intelligence even further.

In this thesis, we have chosen Quoridor as the strategy game to implement our Interfaces on, and analyze results. Designed by Mirko Marchesi, Quoridor stands out as an engaging strategic board game that is played between two or four players. The game is played on a square grid board where the objective of this game is for each player to move their pawn to the opposite side of the board. This game introduces a fascinating twist where a player, in addition to trying to move their pawn through the square grid, additionally has an option to place walls on the grid locations strategically to obstruct the opponent's path. This strategy compels the player to think of their traversal strategy while predicting the opponents strategy as well. Despite its seemingly simple rules, Quoridor demands a unique blend of strategic foresight and the ability to anticipate the moves of opponents and outmaneuver the opponent.

The objective of this thesis is to construct a well-structured framework and user-friendly interfaces that seamlessly integrates AI algorithms into the Quoridor game. The adaptation of our generic interfaces on Quoridor's rule-set will not only enhance our understanding of the game's intricate nuances but also facilitate the creation of an intuitive interface for simulating these AI agents. Furthermore, a comprehensive evaluation will be conducted to identify the suitable parameters for different agents, and several head-to-head games between these agents will help us identify the top-performing one from the set of implemented agents including the Minimax agent, \gls{MCTS} agent and the A-star agent. In addition to this, the project will encompass the creation of a user-friendly interface that empowers players to engage with an AI opponent of their choice visually, thereby bolstering the game's accessibility and inclusivity.

The structure of this thesis is as follows. In Chapter \ref{GameDescription}, we will introduce the formal Notations of the game and based on it, formally explain the rules of the game. In Chapter \ref{GameAnalysis}, we will perform a game analysis from the perspective of game complexity including the state-space complexity and the game tree complexity. In Chapter \ref{background}, we will explain the background and the algorithms behind the implemented AI agents.  In Chapter \ref{Implementation}, we will explain the implementation of the game interface and the AI agents, and provide examples on how one can integrate the said agents in different games. In Chapter \ref{Experiments}, we will simulate the results of the game between the implemented AI agents and finally conclude the thesis in Chapter \ref{Conclusion}.