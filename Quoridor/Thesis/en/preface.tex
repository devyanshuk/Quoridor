\chapter*{Introduction}
\addcontentsline{toc}{chapter}{Introduction}

\ac{AI}, over the past decade, is becoming an integral part of many elements of modern world. In fact, in the recent years, it has become an the main point of evolution and revolution in many of the technologies and industries. From assisted or autonomous driving \citep{Ma2020Artificial}, chat bots \cite{Wu2023ABrief} to gaming \cite{Skinner2010Artificial}, \ac{AI} has been a major revelation in evolving the existing technologies to generating new industry space with the plethora of new use cases.   

\ac{AI} has brought a revolutionary transformation to the gaming world, pushing the boundaries
of what's achievable in both single and multiplayer gaming experiences. In recent years, \ac{AI} has taken center
stage with its remarkable accomplishments in mastering age-old games such as Chess, Go, and many others.
\ac{AI}-driven games now offer users the opportunity to hone and enhance their skills, providing varying difficulty
levels and offering optimal moves to guide players through each step if desired.

Strategy games are a genre of gaming that require planning, often involving various tactics, decision making and execution while under various constraints (e.g., resources). Some of the examples of popular strategy games are Chess, Go, Shogi, Starcraft, etc. Strategy games are unique compared to many other genres of games in the sense that it requires a selection of an optimal move among a subset of sub-optimal moves based on  a plan. In many scenarios, the size of the subset of sub-optimal moves depends on exploration of the game space, simulation and prediction of sequence of the player's and the opponent's moves all while managing certain resources efficiently.

\ac{AI}, due to its suitability towards solving problems involving complex decision making, taking into account sequences of the player's and the opponents possible moves (e.g., game space), outcome prediction or position evaluation all while considering the available resources, have been deemed particularly effective in playing the strategy games. The history of the use of \ac{AI} in strategy games goes back to the last few decades. One of the initial marked impact of \ac{AI} in the strategy gaming came in 1997 when IBM's Deep Blue \cite{Campbell2002Deep} defeated World Chess Champion Garry Kasparov. The influence of \ac{AI} in stratey games was more prominent with the rise and involvement of \ac{AI} into the \ac{RTS} games such as Warcraft and StarCraft \cite{Robertson2014Review} and implementation of \ac{MCTS} in the strategy games such as Go \cite{Huang2011Monte}. Recently, DeepMind's Alpha Go for Go, Alpha Zero for Chess \cite{Silver2017Mastering} and AlphaStar for StarCraft \cite{Team2019Alphastar} have widened the gap between the \ac{AI} and human intelligence even further.


Designed by Mirko Marchesi, Quoridor stands out as an engaging strategic board game that is played between two or four players. The game is playes on a square grid board where the objective of this game is for each player to move their pawn to the opposite side of the board. This game introduces a fascinating twist where a player, in addition to trying to move their pawn through the square grid, additionally have an option to place walls on the grid locations strategically to obstruct the opponent's path. This strategy compels the player to think of their traversal strategy while predicting the opponents strategy as well. Despite its
seemingly simple rules, Quoridor demands a unique blend of strategic foresight and the ability to anticipate the moves of opponents and outmaneuver the opponent.

Compared to some other strategy games such as Chess and Go, Quoridor has not been extensively studied in the literature. In \cite{Glendenning2002MasteringQ}, the author developed an agent for playing Quoridor using genetic algorithm to optimize the weights. The authors in \cite{Mertens2006Quoridor} study the complexity of the algorithm also develop a Quoridor playing agent based on Minimax algorithm. The authors conclude that Quoridor has a similar state-space and game tree complexity as that of the games such as Chess. Likewise, the authors in \cite{Brenner2015Artificial} developed an \ac{MCTS} approach for Quoridor. Recently, in \cite{Iwanaga2022Analysis}, the authors performed an analysis of the game for a miniature 5 by 5 board.

The primary objective of this thesis is to construct a well-structured framework and user-friendly interfaces
that seamlessly integrate AI algorithms into the Quoridor game. The development of AI algorithms customized to
Quoridor's unique rule set will not only enhance our understanding of the game's intricate nuances but also
facilitate the creation of an intuitive interface for simulating these AI agents. Furthermore, a comprehensive
evaluation will be conducted to identify the top-performing AI agent.

In addition to this, the project will encompass the creation of a user-friendly interface that empowers players
to engage with an AI opponent of their choice, thereby bolstering the game's accessibility and inclusivity.

\section {Acknowledged Works}

Quoridor, being a widely popular game, has attracted a fair number of attention from the research community,
resulting in successful AI agent developments. Some of the notable works include:

\begin{itemize}
    \item \textbf{Mastering Quoridor \citep{Glendenning2002MasteringQ}}
    The writer implements and assesses various algorithms like Negamax, Alpha-beta negamax among others.
    Additionally, they utilized a genetic algorithm to refine the weights within a linear weighted evaluation
    function, employing 10 distinct features suggested by the author, some of which include player's position
    towards their goal side, the opponent's position towards their respective goal, the remaining count of
    walls available to the player, etc.
\end{itemize}


In sharp contrast to the aforementioned paper, our thesis takes a distinctly different path by delving deeply
into the architectural aspects of AI. Our approach emphasizes abstraction to the greatest extent possible, with
an eye on facilitating seamless integration into a broad spectrum of games. We prioritize creating an
interface that is adaptable to diverse game environments, setting our research apart from the game-specific
focus of the prior works.



